\makenoidxglossaries
\setacronymstyle{long-short-desc}

\newacronym[description={Design-Verfahren für digitale 3D-Objekte}]{cad}{CAD}{Computer Assisted Design}
\newacronym[description={Laser-basiertes Pulverbett-Verfahren im AM}]{slm}{SLM}{Selective Laser Melting}
\newacronym[description={Nicht mehr in Produktion (engl.)}]{oop}{OOP}{Out-Of-Production}
\newacronym[description={Anderer Begriff für SLM}]{pbf}{PBF}{Powder-Bed-Fusion}
\newacronym[description={Schweißraupen-basiertes Mehrachsiges 3D-Druck-Verfahren}]{lmd}{LMD}{Laser Metal Deposition}
\newacronym[description={Raumfahrtbehörde der Vereinigten Staaten von Amerika}]{nasa}{NASA}{National Aeronautics and Space Administration}
\newglossaryentry{rp}{name={Rapid Prototyping}, description={schnelle Herstellung eines Bauteils oder einer Baugruppe nach 3D-CAD-Daten}}
\newglossaryentry{prusaslicer}{name={PrusaSlicer}, description={Slicer für FDM-Drucker welcher im Rahmen dieser VWA zur visuellen Darstellung genutzt wurde}}
\newglossaryentry{layerline}{
name={Layer-line},
plural={Layer-lines},
description={erkennbare Schichtlinien}}
\newacronym[description={3D-Druck-Verfahren mit UV-härtendem Harz}]{sla}{SLA}{Stereolithographie}
\newacronym[description={3D-Druck-Verfahren mit durch einen Druckkopf ausgepresstem Thermoplastischen Polymeren}]{fdm}{FDM}{Fused Deposition Modeling}
\newacronym[description={Überbegriff des 3D-Drucks}]{am}{AM}{Additive Manufacturing}
\newacronym[description={Beschichtungsverfahren für die Herstellung mikroskopisch dünner, homogener mit Plasma}]{pacvd}{PACVD}{Plasma Assisted Chemical Vapor Deposition}
\newacronym[description={Steuerung von Maschinen über Computer-Programme}]{cnc}{CNC}{Computerized Numerical Control}

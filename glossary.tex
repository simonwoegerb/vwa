\makenoidxglossaries

\newacronym[description={Design-Verfahren für digitale 3D-Objekte}]{cad}{CAD}{\it{Computer Assisted Design}}
\newacronym[description={Laser-basiertes Pulverbett-Verfahren im AM}]{slm}{SLM}{\it{Selective Laser Melting}}
\newacronym[description={Nicht mehr in Produktion (engl.)}]{oop}{OOP}{\it{Out-Of-Production}}


\newacronym[description={Anderer Begriff für SLM}]{pbf}{PBF}{\it{Powder-Bed-Fusion}}
\newacronym[description={Schweißraupen-basiertes Mehrachsiges 3D-Druck-Verfahren}]{lmd}{LMD}{\it{Laser Metal Deposition}}
\newglossaryentry{rp}{name={\it{Rapid Prototyping}}, description={schnelle Herstellung eines Bauteils oder einer Baugruppe nach 3D-CAD-Daten}}
\newglossaryentry{prusaslicer}{name={\it{PrusaSlicer}}, description={Slicer für FDM-Drucker welcher im Rahmen dieser VWA zur visuellen Darstellung genutzt wurde}}
\newglossaryentry{layerline}{
name={\it{Layer-line}},
plural={\it{Layer-lines}},
description={erkennbare Schichtlinien}}
\newacronym[description={3D-Druck-Verfahren mit UV-härtendem Harz}]{sla}{SLA}{\it{Stereolithographie}}
\newacronym[description={3D-Druck-Verfahren mit durch einen Druckkopf ausgepresstem Thermoplastischen Polymeren}]{fdm}{FDM}{\it{Fused Deposition Modeling}}
\newacronym[description={Überbegriff des 3D-Drucks}]{am}{AM}{\it{Additive Manufacturing}}

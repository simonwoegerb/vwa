\documentclass[12pt]{article}
\usepackage{setspace}
\usepackage{siunitx}
\usepackage[german]{babel}
\usepackage[utf8]{inputenc}
\usepackage{graphicx}
\usepackage{csquotes}
\usepackage[acronym]{glossaries}
\usepackage[T1]{fontenc}
\usepackage{float}
\usepackage{blindtext}
\graphicspath{{./img/}}
\hfuzz=5.0002pt
\voffset = 1cm
\setlength{\headheight}{110pt}
\usepackage[
defernumbers=true,
backend=biber,
style=ieee
%style=apa
]{biblatex}
\usepackage{xurl}
\usepackage[hidelinks]{hyperref}
\usepackage[
    left = \glqq{},% 
    right = \grqq{},% 
    leftsub = \glq{},% 
    rightsub = \grq{} %
]{dirtytalk}
\usepackage{tocloft}
\def\hhref#1{\href{#1}{#1}}
\def\ccaption#1#2{\caption[#1\newline#2]{#1}}
\def\it#1{\textit{#1}}
%\DeclareLanguageMapping{german}{german-apa}
\addbibresource{articles.bib}
\addbibresource{web.bib}
\BiblatexSplitbibDefernumbersWarningOff
\usepackage{amsmath}
\onehalfspacing
\usepackage{blindtext}
\usepackage[a4paper, left=3.5cm, right=2.5cm]{geometry}
\usepackage{subfiles}
\makenoidxglossaries

\newacronym[description={Laser-basiertes Pulverbett-Verfahren im AM}]{slm}{SLM}{\it{Selective Laser Melting}}
\newacronym[description={Schweißraupen-basiertes Mehrachsiges 3D-Druck-Verfahren}]{lmd}{LMD}{\it{Laser Metal Deposition}}
\newglossaryentry{rp}{name={\it{Rapid Prototyping}}, description={schnelle Herstellung eines Bauteils oder einer Baugruppe nach 3D-CAD-Daten}}
\newglossaryentry{prusaslicer}{name={\it{PrusaSlicer}}, description={Slicer für FDM-Drucker welcher im Rahmen dieser VWA zur visuellen Darstellung genutzt wurde}}
\newglossaryentry{layerline}{
name={\it{Layer-line}},
plural={\it{Layer-lines}},
description={erkennbare Schichtlinien}}
\newacronym[description={3D-Druck-Verfahren mit UV-härtendem Harz}]{sla}{SLA}{\it{Stereolithographie}}
\newacronym[description={Überbegriff des 3D-Drucks}]{am}{AM}{\it{Additive Manufacturing}}

\defbibfilter{web}{
  type=online
}
\defbibfilter{def}{
  not type=online 
}
\renewbibmacro*{date}{%
  \iffieldundef{year}
    {\bibstring{nodate}}
    {\printdate}
}
\usepackage{fancyhdr}   
\renewbibmacro*{date}{%
	  \iffieldundef{year}
	      {\bibstring{nodate}}
	          {\printdate}}

\begin{document}
\pagenumbering{Roman}
\subfile{title}
%\subfile{abstract}
\pagebreak
%%\subfile{intro}
\pagebreak
\pagenumbering{arabic}
\tableofcontents
\pagebreak
\subfile{chapters/1}
%\subfile{chapters/2}
%\subfile{chapters/3}
\section{Kapazitäten und Grenzen der Maschinen}
\subsection{Geschwindigkeit}
\subsection{Recycling des Exzess-Materials}
\subsection{Druckbare Geometrien (Überhänge)}
\section{Materialien und Eigenschaften}
%Dennoch ist die Vielseitigkeit der Grund warum SLM sich so großer Beliebtheit erfreut. Mit verschiendenen Temperaturen des Lasers lassen sich viele Materialien verarbeiten, wie Fe-C-Stähle mit einem C-Gehalt von maximal \qty{0.3}{\percent}wt, Titan und Aluminium.
\subsection{Einführung Werkstofftechnik/Materialien}
\subsection{Physikalische Eigenschaften}
%\subfile{chapters/4}
\section{Anwendungen in Forschung und Industrie}
\subsection{Forschung}
\subsection{Industrie}
\subsubsection{Reperatur}
%Dies ist besonders praktisch für Firmen, welche auf alte Maschinen angewiesen sind, deren Ersatzteile schon längst außer Produktion sind. Die Anschaffungskosten eines Metall-3D-gedruckten Werkstückes sind zwar enorm hoch, aber dennoch günstiger in vielen Fällen als das Produkt substraktiv herzustellen.
\subsubsection{Leichtbauindustrie}
\pagebreak
\pagebreak

\printnoidxglossary[sort=use, type=\acronymtype]
\printnoidxglossary[sort=use]

\pagebreak
\printbibliography[filter=def]

\pagebreak
\newrefcontext[sorting=none,labelprefix={Web\space}]
\printbibliography[title={Web-Quellen}, filter=web]

\pagebreak
\listoffigures
\end{document}

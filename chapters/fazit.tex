%! TEX root = ../main.tex
\documentclass[../main.tex]{subfiles}
\begin{document}
\section{Fazit}
Zusammenfassend ist festzustellen, dass der Metall-3D-Druck eine Gruppe von Technologien ist, an welchen aktiv geforscht wird. \acrshort{slm} ist die Haupttechnologie in Verwendung und Forschung aufgrund der vielen Möglichkeiten dessen, was damit hergestellt werden kann.
Dennoch verwenden viele Firmen bereits einen Metall-3D-Drucker aufgrund der vielen Vorteile im Kontrast zu subtraktiver Herstellung, wie dem Einsparen von Materialien sowie auch dem Fakt, dass viele Geometrien subtraktiv nicht herstellbar sind. 
Weiterentwicklungen der Technologien ermöglichen in Zukunft revolutionäre Möglichkeiten in Raumfahrt, Luftfahrt, Medizin und in der Wissenschaft als Ganzes, da dadurch höchst stabile Prototypen produzierbar sind.\parencite{Singh2020} 
Weiterführend wäre eine Studie zu den tatsächlichen Ersparnissen höchst spannend, da während der Recherchen solche Daten nicht verfügbar waren. Zudem wäre auch eine praktische werkstofftechnische, destruktive Analyse von 3D-gedruckten Teilen in verschiedenen Orientierungen höchst spannend.

Im Allgemeinen war diese Arbeit eine spannende Erfahrung, besonders im Rahmen des Praktikums an der Fachhochschule Wels, in welchem ich eine \acrshort{slm}-Maschine selbst bedienen und kennenlernen durfte. Dies erlaubte mir ein viel besseres Verständnis der Technologie, welche Literatur nie vermitteln hätte können.


\end{document}

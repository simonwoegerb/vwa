%! TEX root = ../main.tex
\documentclass[../main.tex]{subfiles}
\begin{document}
\section{Stabilität}
\subsection{Orientierung}
Im Selective-Laser-Melting-Verfahren ist die richtige Orientierung essentiell, um stabile Teile zu erhalten.
Ein Winkel von \qty{45}{\degree} führt zu einem Teil, welcher in alle Richtungen die größtmögliche Stärke aufweisst. Dies wird in Abb. dargestellt für einen Teil welcher dem Stressprofil von Abb. folgt.
\begin{figure}[h]
	\centering
	\includegraphics[width=0.5\textwidth]{ss_Flachstab}
	\label{img:ss_1}
	\ccaption{Modell mit generierten Supports}{eigener Screenshot, Lizenzhalter für Materialise Magics 25.02: FH Wels}
\end{figure}
\begin{figure}[h]
	\centering
	\includegraphics[width=0.5\textwidth]{fh_Flachstab}
	\label{img:fh_1}
	\ccaption{Lastenprofil für ein Teil}{Screenshot Studentenaufgabe FH Wels: Ing. Walter Kaindl}
\end{figure}
\end{document}

%! TEX root = ../main.tex
\documentclass[../main.tex]{subfiles}
\begin{document}
\section{Einleitung}
Die vorliegende Arbeit behandelt den Metall-3D-Druck, explizit dabei die \acrlong{slm} und \acrlong{lmd}
-Verfahren. Das Hauptziel meiner Arbeit ist es, die Anwendungen des Metall-3D-Drucks in Forschung und Entwicklung sowie auch in verschiedenen Industrien zu analysieren, als auch die Limitationen und Problematiken der Maschinen festzustellen. Dies findet anhand einer literarischen Studie des aktuellen Forschungsstandes statt. 

Zu Beginn wird das Feld des 3D-Drucks allgemein erklärt im historischen Kontext sowie auch der Weg von der digitalen Entwicklung bis zum fertigen Bauteil. Daraufhin folgt eine Erklärung des Aufbaus der beiden bekanntesten Metall-3D-Druck-Verfahren, \acrshort{slm} und \acrshort{lmd}, sowie auch eine Auseinandersetzung mit dem Thema der \say{Hybridmaschinen}, welche additive Verfahren mit subtraktiven Verfahren wie Drehen und Bohren verbinden, um so die Vorteile beider Arten zu kombinieren. 

Des Weiteren werden die Grenzen und Möglichkeiten der beiden Methoden verglichen und analysiert. Als Letztes folgt eine Auseinandersetzung mit den industriellen und wissenschaftlichen Anwendungen der beiden Verfahren.

Diese Vorwissenschaftliche Arbeit wurde im Rahmen eines Praktikums mit der Fachhochschule Wels verfasst. Geplante Experimente konnten aufgrund von Zeitlimitierungen nicht durchgeführt werden. Im Rahmen des Praktikums wurde hauptsächlich das \acrshort{slm}-Verfahren praktisch behandelt, weswegen diese Arbeit eben dieses fokussiert. Dennoch wurde das \acrshort{lmd}-Verfahren auch mithilfe von Literatur behandelt. 
\end{document}

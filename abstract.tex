%! TEX root = ../main.tex
\documentclass[../main.tex]{subfiles}
\begin{document}
\begin{abstract}
Die vorliegende Arbeit befasst sich mit dem Metall-3D-Druck, spezifisch den \acrfull{slm} und \acrfull{lmd} Verfahren und deren Limitationen beziehungsweise deren Möglichkeiten. 
Zudem werden die am meisten genutzten Materialien und deren hauptsächliche Anwendungsgebiete betrachtet.
Weiters werden die druckbaren Geometrien als auch die Auswirkungen der Orientierung während des Druckpreozesses beider Technologien und die Bedeutung von \it{Supports} im Kontext der Nach-Bearbeitung beleuchtet. 
Im \acrshort{lmd}-Verfahren wird auch auf \it{Hybridmaschinen} eingegangen.
Desweiteren wird auf einige Anwendungsbeispiele von beiden Technologien eingegangen, darunter Luftfahrt, Medizin und Ersatzteil-Produktion. 
\end{abstract}
\end{document}
